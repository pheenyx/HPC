\documentclass[10pt,a4paper]{scrartcl}
\usepackage[utf8]{inputenc}
\usepackage[T1]{fontenc}
\usepackage[ngerman]{babel}
\usepackage{microtype, multicol, marginnote, bera, parskip}
\usepackage{listings, amsmath, amssymb, graphicx, tikz, epic}
\usepackage{stmaryrd} %for lightning arrow
\usepackage{pstricks, pst-node, pst-tree, pdflscape}
\usepackage[babel=true]{csquotes}
\tolerance=2000
\setcounter{secnumdepth}{0}
\usepackage[inner=2.5cm,outer=2.5cm,top=1.5cm,bottom=1.5cm,includeheadfoot]{geometry}

\author{Nils Döring \and Michael Mardaus \and Julian F. Rost \and Sebastian Müller}
\title{High Performance Computing: Sheet 4}

\begin{document}

\maketitle


\section{Question 1}

\begin{lstlisting}{c}
#include <pthread.h>
#include <stdio.h>
#include <stdlib.h>
#include <stdint.h>

#define WORKLOAD 128

int done=0;
int current_job=0;
void* (*jobs[WORKLOAD])( int input );

pthread_mutex_t mutex;
pthread_cond_t awaken;

// input will be the job's id 
void* job( int input )
{
	printf( "%10s(): starting number %d\n", __FUNCTION__, input );
	// SUBSTITUTE YOUR JOB BELOW HERE
	sleep(1);// (simulates "work" being done)
	// SUBSTITUTE YOUR JOB ABOVE HERE
	printf( "%10s(): completed number %d\n", __FUNCTION__, input );
}

void generate( int position );
void* loop( void* id );

int main( int argc, char** argv )
{
	// basic initialization of variables
	int i;
	int t=atoi( argv[1] );
	if( t<1 )
	{
		exit(EXIT_FAILURE);
	} else if( t > WORKLOAD )
	{
		t=WORKLOAD;
	}
	pthread_t threads[t];

	// additional initialization
	pthread_mutex_init( &mutex, NULL );
	pthread_cond_init( &awaken, NULL );

	// create t threads and make them wait for some work
	for( i=0; i<t; ++i )
	{
		pthread_create( &threads[i], NULL, loop, (void*)i );
		printf( "%10s(): created thread %d/%d\n", __FUNCTION__, i, t );
	}

	// generate WORKLOAD many jobs
	for( i=0; i<WORKLOAD; ++i )
	{
		generate( i );
		printf( "%10s(): generated job number %d\n", __FUNCTION__, i );
	}

	// tell all threads to awaken until all jobs have been completed
	while( current_job<WORKLOAD-1 )
	{
		pthread_cond_broadcast( &awaken );
	}
	done=1;

	// join all threads after their work is done
	for( i=0; i<t; ++i )
	{
		pthread_join( threads[i], NULL );
		printf( "%10s(): joined thread %d\n", __FUNCTION__, i );
	}
	printf( "%10s(): waited on %d threads -- done\n", __FUNCTION__, t );

	// cleanup duty -- somebody's gotta do it
	pthread_mutex_destroy( &mutex );
	pthread_cond_destroy( &awaken );
	pthread_exit( NULL );
}

void generate( int position )
{
	jobs[position]=&job;
	pthread_mutex_lock( &mutex );
	printf( "%10s(): signalling job %d is ready\n", __FUNCTION__, position );
	pthread_cond_signal( &awaken );
	pthread_mutex_unlock( &mutex );
}

void* loop( void* id ) 
{
	int my_id=(int)id;

	printf( "%10s(): starting in thread %d\n", __FUNCTION__, my_id );
	pthread_mutex_lock( &mutex );
	while( !done )
	{
		printf( "%10s(): thread %d waiting for work\n", __FUNCTION__, my_id );
		pthread_cond_wait(&awaken, &mutex);
		printf( "%10s(): thread %d received \"awaken\" signal\n", __FUNCTION__, my_id );
		pthread_mutex_unlock( &mutex ); // get read to do the job (in parallel)
		jobs[current_job++]( current_job ); // do the actual job
		pthread_mutex_lock( &mutex ); // wait for another job (serially)
		printf( "%10s(): thread %d going back to sleep\n", __FUNCTION__, my_id );
	}
	pthread_mutex_unlock( &mutex );
	pthread_exit( NULL );
}


\end{lstlisting}



\end{document}
